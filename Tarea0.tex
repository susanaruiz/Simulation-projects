\documentclass{article}
\usepackage[spanish]{babel}
\usepackage[utf8]{inputenc}
\usepackage{fancyhdr}
\usepackage{anysize}
\usepackage[usenames]{color}
\usepackage{booktabs}
\usepackage{pgf,pgfarrows,pgfnodes}
\usepackage{float}
\bibliographystyle{mighelbib}
\addcontentsline{toc}{chapter}{Bibliografía}

\begin{document}

\begin{titlepage}
	\begin{center}
	\huge{\textbf{Tarea 0 Archivo Prueba de latex}}
	\line(1,0) {300}\\
	
	\textsc{ \Large Susana Ruiz Nuñez \\ 2032426} 
	\end{center}
	
\end{titlepage}

\newpage
\section{Tarea 0 Archivo prueba de latex}


Documento para aprender a trabajar en latex. Utilización de diferentes funciones... \cite{Dyckhoff1990}

\subsection{Crear tablas}


\begin{table}[H]
\centering
\caption{Figura Crear tablas}
\begin{tabular}{|c|c|c|}
\hline 
\textbf{Programa} & A & B \\ 
\hline 
\textbf{Python} & 6 horas & 8 horas \\ 
\hline 
\textbf{R} & 0.5 horas & 0.75 horas \\ 
\hline 
\textbf{Latex} & 120 & 160 \\ 
\hline 
\textbf{PDF} & 6.00 & 8.00 \\ 
\hline 
\end{tabular}
\end{table}

\subsection{Numeración}
\begin{itemize}
\item $A_{i}$ = Alfa $i$ 
\item $B_{i}$ = Beta $i$
\item $C_{i}$ = Ceta $i$
\item $D_{i}$ = Di $i$
\item $E_{ij}$ = Extra $j$
\end{itemize}



\section{Bibliografía}

\bibliography{Tarea0biblio}

\end{document}