\documentclass{article}
\usepackage[utf8]{inputenc}
\parskip = 0.75em
\parindent = 10mm
\def\baselinestretch{1}
\usepackage {float}
\usepackage{listings}
\usepackage{subcaption}
\usepackage[usenames]{color}
\usepackage[numbers,sort&compress]{natbib}
\usepackage{multirow, array}
\usepackage[spanish]{babel}
	\deactivatetilden
	\spanishdecimal{.}
	\addto\captionsspanish{\def\tablename{Tabla}}
	\addto\captionsspanish{\def\listtablename{\'Indice de tablas}}

\usepackage{amsmath,amsfonts,amssymb}
	\allowdisplaybreaks[4]
\usepackage{graphicx}
	\graphicspath{{Figuras/}}
\usepackage[clearempty,pagestyles]{titlesec}
\usepackage{anysize}

\def\baselinestretch{1.5}
\papersize{27.9cm}{21.5cm} 
\marginsize{2cm}{2cm}{1cm}{1cm}
\usepackage{multicol}

\begin{document}

	\begin{center}
	\huge{\textbf{Simulación de empaquetamiento de objetos poligonales}}\\
	
	\textsc{ \Large Susana Ruiz Nuñez}
	\end{center}


\begin{center}\rule{0.9\textwidth}{0.1mm} \end{center}
\begin{abstract}
	\normalsize Un artículo suele empezarse con un resumen.\\ \\
	Palabras clave: Empaquetamiento, simulación.

	\vspace*{0.5cm}
\end{abstract}

\begin{multicols}{2}
\section{Introducción} 
En este proyecto

\section{Antecedentes}
Se conoce que incluso la versión unidimensional del problema de encontrar el óptimo al uso de un recurso dado, el problema clásico de la mochila, pertenece a la clase de NP-difícil de problemas de optimización. Por esta razón, la mayor parte del trabajo relacionado con problemas de corte y embalaje emplean enfoques heurísticos. No obstante, el desarrollo de métodos de solución exactos es una tarea importante para ampliar la gama de casos óptimos solucionables \cite{leao}.

El problema de empaquetamiento en un rectángulo con dimensiones (largo o/y ancho) abiertas es conocido como “Bin Packing Problem”. En este problema un conjunto de objetos (por ejemplo, polígonos convexos o círculos) se debe cortar de una faja o placa de forma rectangular. Los objetos pueden ser orientados libremente. Existen dos formas a mencionar; cuando las placas de diseño necesitan ser producidas, o se encuentran ya disponibles en stock. El objetivo es minimizar el área de los rectángulos de diseño. Las placas de diseño están sujetos a los límites inferior y superior de sus anchos y longitudes \cite{roma}.


\section{Trabajos relacionados}

Se encuentran publicaciones sobre el empaquetamiento de círculos en un rectángulo que minimice el área del mismo \cite{luba}. En \cite{benn} una serie de óptimos locales se proponen para minimizar el área de un rectángulo que contiene un conjunto de círculos.
En \cite{kall} se consideran el problema de empaque de círculos en rectángulos y otras formas geométricas. Otro trabajo que vale la pena mencionar es \cite{hifi} donde calculan el perímetro mínimo en rectángulos que encierran círculos congruentes no superpuestos. Otro enfoque interesante está propuesto en \cite{arau} donde se formuló el problema de empacar un conjunto de círculos de diferentes tamaños dentro del área del cuadrado más pequeño posible como un problema de programación no lineal (NLP) y estableció las condiciones de primer orden de optimalidad.

\section{Modelo propuesto}


\section{Implementación de la simulación}


\section{Experimentos (diseño, resultados y discusión)}

\section{Conclusiones}

\section{Trabajo futuro}

\bibliography{Bibliografia}
\bibliographystyle{plainnat}
\end{multicols}
\end{document} 
